
\documentclass{article}
\pdfpagewidth=8.5in
\pdfpageheight=11in
\usepackage{ijcai20}

% Use the postscript times font!
\usepackage{times}
\renewcommand*\ttdefault{txtt}
\usepackage{soul}
\usepackage{url}
\usepackage[hidelinks]{hyperref}
\usepackage[utf8]{inputenc}
\usepackage[small]{caption}
\usepackage{graphicx}
\usepackage{amsmath}
\usepackage{booktabs}
\urlstyle{same}


\title{Hand Gesture Recognition on Arduino Using Time Series Classification}

\author{
Matthew Lipski\\
Supervisors: Mingkun Yang\and
Ran Zhu
\affiliations
EEMCS, Delft University of Technology, The Netherlands\\
\emails
m.s.lipski@student.tudelft.nl,
m.yang-3@tudelft.nl,
r.zhu-1@tudelft.nl
}

\begin{document}

\maketitle

%\begin{abstract}
The aim of this template is to make it more clear what is expected from you. 
\textbf{It is by no means required to follow this exact same structure.}
The abstract should be short and give the overall idea:
what is the background, the research questions, what are your contributions, and what are the main conclusions.
It should be readable as a stand-alone text (preferably no references to the paper or to outside literature).
\end{abstract}

\setlength{\parskip}{\baselineskip}%
\setlength{\parindent}{0pt}%

\section{Introduction}\label{sec:introduction}
\subsection{Research Overview}\label{subsec:research-overview}
Traditionally, physical buttons have been by far the most common way for users to interact with electronic devices in public settings, whether these are coffee machines, elevator panels, or train ticket machines.
However, the concern of disease transmission has become increasingly prevalent in recent years due to the COVID-19 pandemic, making it enticing to develop an alternative solution which does not require touch.
One such solution is the use of hand gestures to interact with public devices instead.
By performing hand motions such as swiping and tapping, users can effectively and intuitively interact with electronic devices with minimum risk of disease transmission.

However, there are a few key challenges to this approach which stand in the way of it replacing physical buttons in real-world applications.
\begin{enumerate}
    \item Additional hardware is needed to detect the positions of a user's hand while performing a gesture.
    The data output by this hardware is likely to be low in resolution since system costs should be minimized.
    \item Additional software must be implemented to recognize gestures based on hand position over time.
    While buttons are just simple digital inputs, one gesture can also be performed differently between different users, yet the system must be able to accurately classify it regardless.
    \item Gesture recognition must be done in real-time.
    Since the process of classifying gestures can be quite complex, the latency introduced by this may be significant.
    However, the user should not perceive any lag while using the system for a positive experience.
\end{enumerate}

\subsection{Research Question}\label{subsec:research-question}
Given the challenges in developing a neural network for gesture recognition, the goal of this paper can be summarized with the following research question:

\textbf{'Which neural network architecture is most appropriate for recognizing hand gestures on an Arduino Nano 33 BLE, using 3D-formatted data from OPT101 photodiodes?'}

This can then be segmented into the following sub-questions:
\begin{enumerate}
    \item Which neural network architectures produce the highest accuracy for hand gesture recognition?
    \item Which neural network architectures produce the lowest inference latency for hand gesture recognition?
    \item What is the minimum acceptable accuracy for recognizing hand gestures on an Arduino Nano 33 BLE?
    \item What is the maximum acceptable inference latency for recognizing hand gestures on an Arduino Nano 33 BLE?
    \item How can 3D-formatting data be exploited for better gesture recognition performance?
\end{enumerate}

\subsection{Contributions}\label{subsec:contributions}
This research overcomes these challenges by using data from OPT101 photodiodes, which is fed into a neural network to recognize gestures with high accuracy and low latency on an Arduino Nano 33 BLE microcontroller.
It is also part of a larger project which integrates this neural network into a full gesture recognition system, which is elaborated on in section 3.2.

Similar research which involves using photodiodes and machine learning to recognize hand gestures has already been conducted, but this paper improves on existing solutions in a number of ways:
\begin{enumerate}
    \item A high level of accuracy is maintained with fewer photodiodes, and therefore fewer model input features, than existing solutions.
    \item The data from photodiodes is 3D-formatted, which better preserves temporal information and improves recognition accuracy.
    An explanation for what 3D-formatting involves can be found in section~\ref{subsec:3d-formatted-data}.
    \item The neural network used to classify hand gestures is smaller in size than competing solutions, leading to reduced system latency.
\end{enumerate}

A discussion regarding existing research in this field is found in section~\ref{sec:related-work}, which provides more context to these improvements.

%The planned use case for this technology is hands-free navigation of menus which has especially gained relevance due to the restrictions imposed during the COVID-19 pandemic, which has showed that gestures can be an appealing alternative to physical buttons in social settings.
%The most important existing literature for this research comes from Pete Warden and Daniel Situnayake, who are pioneers in the field of embedded AI and authors of the book "TinyML: Machine Learning with TensorFlow Lite on Arduino and Ultra-Low-Power Microcontrollers"\cite{warden2020tinyml}, as well as Qing Wang and Marco Zuniga, who have laid the groundwork for embedded AI specifically in the context of hand gesture recognition\cite{10.1145/3412449.3412551}.
%The research conducted for this paper expands on existing work in this exact way, by using a lower power microcontroller and significantly fewer photo diodes than in the solution created by Wang and Zuniga, while maintaining or improving the classification accuracy of hand gestures.


%\section{Methodology or Problem Description}
Choose one that fits your research best:
\subsection{Methodology}
Typically in general research articles, the second section contains a description of the research methodology, explaining what you, the researcher, is doing to answer the research question(s), and why you have chosen this method.
For purely analytical work this is a description of the data collection or experimental setup on how to test the hypothesis, with a motivation.
In any case this section includes references to necessary background information.
For a survey paper this includes the method of how you arrived at the set of papers included in the survey.

\subsection{Formal Problem Description}
For some types of work in computer science the methodology is standard: analyze the problem (e.g., make assumptions and derive properties), present a new algorithm and its theoretical background, proving its correctness, and evaluate unproven aspects in simulation.
Then an explanation of the methodology is often omitted, and the setup of the evaluation is part of a later section on the evaluation of the ideas.\footnote{This already shows that there is no single outline to be given for all papers.}
In this case, explain relevant (background) concepts, theory and models in this section (with references) and relate them to your research question.
Also this section then typically contains a more precise, formal description of the problem.

Do not forget to give this section another name, for example after the problem you are solving.

%
%\section{Your contribution (replace this section title by something more informative)}
In computer science typically the third section contains an exposition of the main ideas, for example the development of a theory, the analysis of the problem (some proofs), a new algorithm, and potentially some theoretical analysis of the properties of the algorithm.

Do not forget to give this section another name, for example after the method or idea you are presenting.

Some more detailed suggestions for typical types of contributions in computer science are described in the following subsections.


\subsection*{Experimental work}
In this case, this section will mostly contain a description of the methods/algorithms you will be comparing. Although not all methods need to be described in detail (providing appropriate references are available), make sure that you reveal sufficient details to a reader not familiar with these methods to: a) obtain a high-level understanding of the method and differences between them, and b) understand your explanation of the results/conclusions.

\subsection*{Improvement of an idea}
In this case, you would need to explain in detail how the improvement works. If it is based on some observation that can be proven, this is a good place to provide that proof (e.g., of the correctness of your approach). 

\subsection*{Literature survey}
If your contribution is a literature survey, then the organization of these ``middle'' sections very much depends on the way you want to present/organize the literature you are discussing.
First try to cluster papers that are similar in some aspect. Then think of how these clusters are related, from that you can think of a good order to discuss these clusters; this is sometimes called a bottom-up approach to writing a paper.

In addition, you may try to think about the organization of the literature from a top-down perspective: try to ``take a step back'' and think about the field and what important questions/variants are and build a hierarchical categorization of the field.

Make clear what your contribution is here: a new organization of the literature, identification of open problems/challenges, new parallels/generalizations, a table with pros/cons of different methods, etc.\ 


%
%\section{Experimental Setup and Results}
As discussed earlier, in many sciences the methodology is explained in section 2 and this section only discusses the results. 
However, in computer science, most often the details of the evaluation setup are described here first (simulation environment, etc.).
Very important is that any skilled reader would be able to reproduce this setup and then obtain the same results.

Then, results are reported in an accessible manner through figures (preferably with captions that allow them to be understood without going through the whole text), observations are made that clearly follow from the presented results.
Conclusions are drawn that follow logically from the previous material.
Sometimes the conclusions are in fact hypotheses, which in turn may give rise to new experiments to be validated.

You may want to give this section another name.
%
%\section{Responsible Research}\label{sec:responsible-research}
All code used for this research is open source can be found at https://github.com/matthewlipski/research\_project.
Due to the inherent randomness present when training neural networks, the results presented in this paper may not be completely reproducible.
However, using the same code and hardware, it should be possible for anyone to reach the same conclusions based on their own findings.
The code being open source also allows others to find potential flaws in it, allowing it to be worked on and used for future research.

The work conducted for this study was also heavily focused on experimentation rather than reviewing literature.
Naturally, research into existing work was still required to provide a starting point for testing and find which neural network architectures would be most suitable for recognizing gestures, as well as find potential improvements to existing work.
However, this emphasis on testing and experimentation rather than reading literature means that the testing methodology falls under the under most scrutiny, and that ensuring the results of this research are reproducible is paramount.
This is why much of the paper is devoted to explaining the system and testing setup.
Although it may seem overly detailed at times, it is crucial to go over the methodology in depth in order for the results presented in this study to be reproducible.

%
%\section{Discussion}
Results can be compared to known results and placed in a broader context.
Provide a reflection on what has been concluded and how this was done.
Then give a further possible explanation of results.

You may give this section another name, or merge it with the one before or the one hereafter.
%
%\section{Conclusions and Future Work}
Briefly summarize the (main) research question(s).
Provide your conclusions, the answers to the research question(s).
Make statements.
Highlight interesting elements, contributions.

Discuss open issues, possible improvements, and new questions that arise from this work; formulate recommendations for further research.

Ideally, this section can stand on its own: it should be readable without having read the earlier sections and accessible to anyone with a bachelor degree in Computer Science.

%
%\appendix
\section{Some further guidelines that go without saying (right?)}

\begin{itemize}
\item Read the manual for the Research Project. (See e.g.\ the instructions on the maximum length: less is more!)
\end{itemize}

\subsection{Reference use}
\begin{itemize}
\item use a system for generating the bibliographic information automatically from your database, e.g., use BibTex and/or Mendeley, EndNote, Papers, or \ldots
\item all ideas, fragments, figures and data that have been quoted from other work have correct references
\item literal quotations have quotation marks and page numbers
\item paraphrases are not too close to the original
\item the references and bibliography meet the requirements
\item every reference in the text corresponds to an item in the bibliography and vice versa
\end{itemize}

\subsection{Structure}
Paragraphs
\begin{itemize}
\item are well-constructed
\item are not too long: each paragraph discusses one topic
\item start with clear topic sentences
\item are divided into a clear paragraph structure
\item there is a clear line of argumentation from research question to conclusions
\item scientific literature is reviewed critically
\end{itemize}

\subsection{Style}
\begin{itemize}
\item correct use of English: understandable, no spelling errors, acceptable grammar, no lexical mistakes 
\item the style used is objective
\item clarity: sentences are not too complicated (not too long), there is no ambiguity
\item attractiveness: sentence length is varied, active voice and passive voice are mixed
\end{itemize}

\subsection{Tables and figures}
\begin{itemize}
\item all have a number and a caption
\item all are referred to at least once in the text
\item if copied, they contain a reference
\item can be interpreted on their own (e.g. by means of a legend)
\end{itemize}


\bibliographystyle{plain}
\bibliography{references}

%A rule of thumb for dealing with the literature is the following: scan about 10--20 contributions: read title, abstract, part of introduction and conclusions; categorize contribution; some of these are studied in more depth: completely read about 5 conference papers or equivalent (summarize contribution in own words); of which studied in-depth about 2 conference papers (the student is able to explain in detail and criticize contributions). This may result in 5--20 references, possibly even more if the project is a literature study.


\end{document}

