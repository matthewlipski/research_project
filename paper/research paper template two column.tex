
\documentclass{article}
\pdfpagewidth=8.5in
\pdfpageheight=11in
\usepackage{ijcai20}

% Use the postscript times font!
\usepackage{times}
\renewcommand*\ttdefault{txtt}
\usepackage{soul}
\usepackage{url}
\usepackage[hidelinks]{hyperref}
\usepackage[utf8]{inputenc}
\usepackage[small]{caption}
\usepackage{graphicx}
\usepackage{amsmath}
\usepackage{booktabs}
\urlstyle{same}


\title{Hand Gesture Recognition on Arduino Using Machine Learning and Ambient Light}

\author{
Matthew Lipski\\
Supervisors: Mingkun Yang\and
Ran Zhu
\affiliations
EEMCS, Delft University of Technology, The Netherlands\\
\emails
m.s.lipski@student.tudelft.nl,
m.yang-3@tudelft.nl,
r.zhu-1@tudelft.nl
}

\begin{document}

\maketitle

%\input{sections/abstract}

\setlength{\parskip}{\baselineskip}%
\setlength{\parindent}{0pt}%

\section{Introduction}\label{sec:introduction}
\subsection{Research Overview}\label{subsec:research-overview}
Traditionally, physical buttons have been by far the most common way for users to interact with electronic devices in public settings, whether these are coffee machines, elevator panels, or train ticket machines.
However, the concern of disease transmission has become increasingly prevalent in recent years due to the COVID-19 pandemic, making it enticing to develop an alternative solution which does not require touch.
One such solution is the use of hand gestures to interact with public devices instead.
By performing hand motions such as swiping and tapping, users can effectively and intuitively interact with electronic devices with minimum risk of disease transmission.

However, there are a few key challenges to this approach which stand in the way of it replacing physical buttons in real-world applications.
\begin{enumerate}
    \item Additional hardware is needed to detect the positions of a user's hand while performing a gesture.
    The data output by this hardware is likely to be low in resolution since system costs should be minimized.
    \item Additional software must be implemented to recognize gestures based on hand position over time.
    While buttons are just simple digital inputs, one gesture can also be performed differently between different users, yet the system must be able to accurately classify it regardless.
    \item Gesture recognition must be done in real-time.
    Since the process of classifying gestures can be quite complex, the latency introduced by this may be significant.
    However, the user should not perceive any lag while using the system for a positive experience.
\end{enumerate}

\subsection{Research Question}\label{subsec:research-question}
Given the challenges in developing a neural network for gesture recognition, the goal of this paper can be summarized with the following research question:

\textbf{'Which neural network architecture is most appropriate for recognizing hand gestures on an Arduino Nano 33 BLE, using 3D-formatted data from OPT101 photodiodes?'}

This can then be segmented into the following sub-questions:
\begin{enumerate}
    \item Which neural network architectures produce the highest accuracy for hand gesture recognition?
    \item Which neural network architectures produce the lowest latency for hand gesture recognition?
    \item What is the minimum acceptable accuracy for recognizing hand gestures on an Arduino Nano 33 BLE?
    \item What is the maximum acceptable latency for recognizing hand gestures on an Arduino Nano 33 BLE?
    \item How can 3D-formatting data be exploited for better gesture recognition performance?
\end{enumerate}

\subsection{Contributions}\label{subsec:contributions}
This research overcomes these challenges by using data from OPT101 photodiodes, which is fed into a neural network to recognize gestures with high accuracy and low latency on an Arduino Nano 33 BLE microcontroller.
It is also part of a larger project which integrates this neural network into a full gesture recognition system, which is elaborated on in section 3.2.

Similar research which involves using photodiodes and machine learning to recognize hand gestures has already been conducted, but this paper improves on existing solutions in a number of ways:
\begin{enumerate}
    \item A high level of accuracy is maintained with fewer photodiodes, and therefore fewer model input features, than existing solutions.
    \item The data from photodiodes is 3D-formatted, which better preserves temporal information and improves recognition accuracy.
    An explanation for what 3D-formatting involves can be found in section~\ref{subsec:3d-formatted-data}.
    \item The neural network used to classify hand gestures is shallower than competing solutions, leading to reduced system latency while maintaining a high level of accuracy.
\end{enumerate}

A discussion regarding existing research in this field is found in section~\ref{sec:related-work}, which provides more context to these improvements.

%The planned use case for this technology is hands-free navigation of menus which has especially gained relevance due to the restrictions imposed during the COVID-19 pandemic, which has showed that gestures can be an appealing alternative to physical buttons in social settings.
%The most important existing literature for this research comes from Pete Warden and Daniel Situnayake, who are pioneers in the field of embedded AI and authors of the book "TinyML: Machine Learning with TensorFlow Lite on Arduino and Ultra-Low-Power Microcontrollers"\cite{warden2020tinyml}, as well as Qing Wang and Marco Zuniga, who have laid the groundwork for embedded AI specifically in the context of hand gesture recognition\cite{10.1145/3412449.3412551}.
%The research conducted for this paper expands on existing work in this exact way, by using a lower power microcontroller and significantly fewer photo diodes than in the solution created by Wang and Zuniga, while maintaining or improving the classification accuracy of hand gestures.


\section{Background}\label{sec:background}
\subsection{Machine Learning \& Artificial Neural Networks}\label{subsec:machine-learning-and-neural-networks}
Machine learning is a sub-field of artificial intelligence which "provides learning capability to computers
without being explicitly programmed"~\cite{Alzubi_2018}.
More specifically, machine learning allows computers to "learn" patterns and trends from existing data in order to make accurate predictions on new data, without any human input.

Artificial neural networks (ANNs) are a type of machine learning model which draw inspiration from the human brain~\cite{Wang2003}.
These types of models feature "neurons" organized into densely connected layers.
The first layer, which takes in the input data, is known as the input layer while the final layer, which yields the processed output data, is called the output layer.
In between these is at least one hidden layer and in general, having more hidden layers and more neurons per hidden layer allows a neural network to approximate increasingly complex functions.

\begin{figure}[h]
    \centering
    \captionsetup{justification=centering}
    \includegraphics[width=\linewidth]{figures/ann}
    \caption{Visualization of a generic ANN with 8 input neurons and 4 output neurons, as well as a single hidden layer with 11 neurons.}
    \label{fig:ann}
\end{figure}

Each connection between two neurons has an associated weight and bias, which determine how much the output from the source neuron affects the output of the destination neuron.
These values change as the model "learns" from existing data, causing the network's performance to gradually improve.
In addition to this, each neuron passes the combined input of all neurons in the previous layer through a non-linear activation function, which is what allows neural networks to effectively model any function possible~\cite{Wang2003}.

Although ANNs are no longer considered state-of-the-art, understanding the purposes of neurons, neuron connections, layers, and activation functions remains useful, as these still form the building blocks of any neural network architecture.

\subsection{Neural Networks on Embedded Hardware}\label{subsec:neural-networks-on-embedded-hardware}
Machine learning, and specifically neural networks, have traditionally been restricted to the realms of high-performance and in turn, high power devices~\cite{8342164}.
Unfortunately, this means that it has been previously impractical to use these technologies with embedded hardware such as microcontrollers, as they lack the memory \& performance to run inference on neural networks locally, while using a remote processor for inference is not always feasible.

Recent advances into machine learning model compression and optimization have changed this, allowing deep neural networks with multiple hidden layers to be run on devices even powered by coin batteries.
The most prominent development in this field has been TensorFlow Lite, which is a multi-language framework used to optimize existing neural networks such that they can be ran on mobile devices~\cite{MLSYS2021_d2ddea18}.
The key optimization introduced by TensorFlow Lite is quantization, which allows converts all point weights and activation functions in a network to 8-bit integers instead of 32-bit floating points, resulting in a tremendous improvement to the size of the neural network in memory as well as its runtime.

However, mobile devices still have considerably more processing power and hardware capabilities than embedded devices, which is an issue that led to the development of TensorFlow Lite for Microcontrollers in 2018~\cite{MLSYS2021_d2ddea18}.
TensorFlow Lite for Microcontrollers is a fork of TensorFlow Lite which further optimizes models such that they can be run on devices with extremely limited resources by using exclusively C/C++ and cutting down massively on dependencies.

\subsection{Hand Gesture Data}\label{subsec:hand-gesture-data}
In this research, neural networks are used to detect gestures, but to do this, data from some sensor(s) must be fed into the network.

There are a variety of sensors which could be used to record hand gestures and provide this data.
One of these is an accelerometer attached to the user's wrist (typically from a smartwatch) to track the direction and acceleration of the their hand movements~\cite{4912759}.
Another option is using a depth/range camera to record the user's hand, which provides a huge amount of information but in turn requires a computationally intensive video processing pipeline to make sense of the data~\cite{article}.
Photodiodes can also be used, which are sensors that output a signal which increases with the amount of light that hits them.
This means they can track the shadows cast by the user's hand under ambient light, therefore making it possible to recognize which gesture is being performed~\cite{8947919}.
This project uses photodiodes for their much lower monetary and computational cost compared to cameras as well as the fact that they don't require the user to wear a device on their wrists, unlike accelerometers.

\subsection{3D-Formatted Data}\label{subsec:3d-formatted-data}
The term "3D-formatted data" is specific to this research, and must be explained to understand the choice of neural network architectures tested.
This is best done by comparing it to "2D-formatted data".

2D-formatted data can be represented as an image, with some horizontal resolution $x$ and vertical resolution $y$.
In this research, each photodiode outputs values at a predetermined sampling rate over the course of a gesture.
This means that data can be formatted as a 2D image in which $x$ is the number of photodiodes used and $y$ is the number of total samples received from any of the photodiodes.
Therefore, the value of each "pixel" in the image represents a reading from a single photodiode at a single point in time.

3D-formatted data can meanwhile be thought of as a video, which splits this 2D-image into a sequence of $n$ frames, as shown in figure~\ref{fig:3d-data}\@.
3D-formatting is generally more appropriate when the data is sensitive to time, i.e.\ when data points should be considered in a specific sequence.
Therefore, by 3D-formatting the photo diode data, lower error rates should be achievable as temporal information from the photo diodes isn't lost.

\begin{figure}[h]
    \centering
    \captionsetup{justification=centering}
    \includegraphics[width=\linewidth]{figures/3d_data}
    \caption{Visualization of 2D photodiode data after being 3D-formatted into 5 frames.}
    \label{fig:3d-data}
\end{figure}
%
\section{System Overview}\label{sec:system-overview}
\subsection{Full Gesture Recognition System}\label{subsec:full-gesture-recognition-pipeline}
This research focuses on finding an appropriate neural network architecture to perform gesture recognition on a microcontroller, but it is only part of a larger project to create an entire gesture recognition system/pipeline.
The creation of this pipeline is composed of the following tasks:
\begin{enumerate}
    \item Optimizing the number and placement of OPT101 photo diodes.
    \item Pre-processing data from photo diodes.
    \item Creating an appropriate dataset for training a neural network to recognize gestures.
    \item Finding an appropriate neural network architecture on the created dataset and ensuring gestures can be recognized in real-time on an Arduino Nano 33 BLE\@.
\end{enumerate}

The research presented in this paper aims to complete task 4.
Although tasks 1--3 were completed by other project group members and are beyond the scope of this research, they are worth mentioning to provide some context regarding the rest of the gesture recognition system.
Due to the findings from these tasks, the final system uses 3 photodiodes and can recognize 10 different gestures, while each gesture is composed of 100 samples from each photodiode.
This is relevant for task 4, as it means that whatever neural network is implemented must use a 2D array of size 3 by 100 (split into \textit{n} frames after 3D-formatting) as an input feature and be able to distinguish between 10 output classes, as illustrated in figure~\ref{fig:system}.

\begin{figure}[h]
    \centering
    \captionsetup{justification=centering}
    \includegraphics[width=\linewidth]{figures/gestures}
    \caption{Illustrations of the 10 different gestures that the system can recognize.}
    \label{fig:gestures}
\end{figure}

\begin{figure}[h]
    \centering
    \captionsetup{justification=centering}
    \includegraphics[width=\linewidth]{figures/system_advanced}
    \caption{Visualization of the input features \& output classes using a generic artificial neural network as an example.}
    \label{fig:system}
\end{figure}

\subsection{System Caveats}\label{subsec:system-caveats}
The overarching goal of the project that this research contributes to, is the creation of a full gesture recognition pipeline, which presents some issues when considering the fact that each part of this pipeline was developed in parallel.
In reality, it would make much more sense to complete each step of the pipeline sequentially, as the performance of later parts of the pipeline relies on the performance of previous parts.
To put this in the context of the gesture recognition system, it is impossible to train a neural network to recognize gestures without first having a dataset to train it on.
However, the creation of that dataset relies on photodiode count and placement, as well as sampling rate and pre-processing, being finalized.
If they change after the dataset is completed, it will not be representative of real-world data, leading to poor gesture classification performance from the neural network trained on it.

\subsection{Dataset}\label{subsec:dataset}
Having a varied, expansive, and representative dataset is crucial for training a machine learning with high real-world accuracy.
Fortunately, a dataset for recognizing gestures using photodiode data was done by another member of the project group, as mentioned in section~\ref{subsec:full-gesture-recognition-pipeline}.

Unfortunately, because the allotted time for the project meant that all group members had to work in parallel, as stated in section~\ref{subsec:system-caveats}, the neural networks evaluated in this paper had to be trained on a dataset that is not final.
Specifically, the dataset used in this research is not passed through the data pre-processing stage of the system.
This means that the neural networks presented in this paper were trained on raw data from the photodiodes, whereas real-world data on the final system would include this pre-processing step.
This leads to accuracy measurements being lower than they could be, as there is noise and other artifacts in the raw photodiode data.

The dataset used contains 5 repetitions of each of the 10 gestures per hand across 48 participants, leading to 4800 total data instances.
Each instance is made up of a 5 second window during which a gesture is performed over the photodiodes, with a sampling rate of 20Hz for a total of 100 samples per photodiode.
Although the dataset contains instances from a variety of environments and lighting setups, these are mostly indoor locations as this is the planned use case for the system.
The dataset was also mostly recorded on the TU Delft campus, meaning the demographic of participants is somewhat skewed.
Most notably, the dataset contains substantially more instances of males compared to females, and more right-handed participants than left-handed.
However, this is not expected to have a large impact on the performance on any neural networks trained on this dataset.
%
\section{Model Design}\label{sec:model-design}
\subsection{Architectures Tested}\label{subsec:architectures-tested}
Recognizing hand gestures based on photodiode data can be thought of as a time-series classification task, which is a type of problem that recurrent neural networks (RNNs) are especially well suited for.
RNNs differ from conventional neural networks as they do not process the input in its entirety, but instead process each time step or sample from the input sequentially.
Each time step is used to update a "hidden state", which is fed back into the RNN along with the next time step.
This gives RNNs the ability to exploit the time-sensitive nature of time-series data, as each time step also has temporal data associated with it.
This desirable property means that RNNs were the first type of network tested.

Although RNNs are highly suitable for time-series classification, they suffer from the "vanishing gradient problem", which has already been overcome by long short-term memory cells (LSTMs) and gated recurrent units (GRUs).
To briefly explain this problem, the weight of the hidden state of a time step in an RNN tends to exponentially decrease for future time steps.
This means that in practice, RNNs tend to ignore if earlier time steps are out of order, which limits performance for long data sequences (CITATION NEEDED).
LSTMs and GRUs solve this issue using so-called "gates", which determine what contents of the previous hidden state should be kept and what contents of the current hidden state should be propagated to the next time step, which largely mitigates the vanishing gradient problem as only relevant information is kept, therefore greatly reducing the rate at which hidden state weights decay.
Given that LSTMs and GRUs are should effectively guarantee better classification performance than RNNs due to their inherent advantages, these were also investigated.

One issue of using LSTMs and GRUs, however, is that they only accept a single data sequence with multiple channels as input.
This is relevant as with 3D-formatted data, each time step is a 2D frame, which can't be used as input for these types of neural networks as they only expect a single value from each channel per time step, i.e.\ a 1D array.
Therefore, each frame has to be flattened first, which removes some spacial data.
This issue can be solved by using convolutional LSTMs, which use convolutional kernels in place of traditional neurons, treating each time step as an image, which is exactly how 2D frames should be processed.
Due to this, both 1D and 2D convolutional LSTMs were investigated for this paper, with the main difference being that the former performs 1D convolutions on each row of the frame individually, therefore keeping the data from each photodiode separate, while the latter performs 2D convolutions on the entire frame.

Finally, transformer encoders were also tested for this project as a novel alternative to RNN based architectures.
Transformers are typically used in language translation and are made up of two parts, an encoder and a decoder.
The encoder converts the input into an internal representation, which can then be manipulated and decoded back into the the same data format as the input.
However, by using only the encoder, the internal representation of the input can instead by directly mapped to a gesture, therefore making this architecture suitable for classification problems.
Transformers also use a concept called self-attention which identifies which elements in a sequence carry most semantic value based on the other elements of the sequence, which is fundamentally different to the hidden state mechanism used by RNNs and their derivatives.

%
%\section{Responsible Research}\label{sec:responsible-research}
All code used for this research is open source can be found at https://github.com/matthewlipski/research\_project.
Due to the inherent randomness present when training neural networks, the results presented in this paper may not be completely reproducible.
However, using the same code and hardware, it should be possible for anyone to reach the same conclusions based on their own findings.
%
%\input{sections/discussion}
%
%\input{sections/conclusion_and_future_work}
%
%\input{sections/appendix}

\bibliographystyle{plain}
\bibliography{references}

%A rule of thumb for dealing with the literature is the following: scan about 10--20 contributions: read title, abstract, part of introduction and conclusions; categorize contribution; some of these are studied in more depth: completely read about 5 conference papers or equivalent (summarize contribution in own words); of which studied in-depth about 2 conference papers (the student is able to explain in detail and criticize contributions). This may result in 5--20 references, possibly even more if the project is a literature study.


\end{document}

