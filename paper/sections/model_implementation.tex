\section{Model Implementation}\label{sec:model-implementation}
\subsection{Training Code}\label{subsec:training-code}
The implementation and testing of each neural network architecture was done using the TensorFlow Python library and high-level Keras API\@.
While the dataset is split into training and validation sets when evaluating architecture performance, the final models used on the Arduino are trained on all available data, after which they are converted into TensorFlow Lite models.

\subsection{Inference Code}\label{subsec:inference-code}
To run the TensorFlow Lite models created during training on an Arduino, they must first be converted into C files.
This was done using the method detailed in chapter 4.5 of "TinyML Machine Learning with TensorFlow Lite on Arduino and Ultra-Low-Power Microcontrollers"~\cite{warden2020tinyml}.
TODO: ADD SPECIFICS REGARDING FINAL IMPLEMENTATION ON ARDUINO.