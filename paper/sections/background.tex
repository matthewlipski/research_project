\section{Background}\label{sec:background}
\subsection{Machine Learning \& Neural Networks}\label{subsec:machine-learning-&-neural-networks}
Machine learning is a sub-field of artificial intelligence which gives a machine the capacity to "imitate the way that humans learn, gradually improving its accuracy" (https://www.ibm.com/cloud/learn/machine-learning).
More specifically, machine learning models "learn" patterns and trends from existing data in order to make accurate predictions on new, unseen data.

Neural networks, meanwhile, are a type of machine learning model which draws inspiration from the human brain.
These types of models feature "neurons" organized into layers, with each layer typically being densely connected to the next.
However, the structure of different neural networks can vary massively, but their main advantage over simpler models is their flexibility and improved ability to deal with highly non-linear data.

\subsection{Machine Learning on Embedded Hardware}\label{subsec:machine-learning-on-microcontrollers}
Machine learning and artificial intelligence have traditionally been restricted to the realms of high-performance and in turn, high power devices.
Unfortunately, this means that it has been previously unfeasible to use these technologies with embedded hardware as it lacks the performance to run machine learning models locally, and lacks the dedicated power to be able to transmit sensor data to a remote processor.
However, recent advances into machine learning model compression and optimization have changed this, allowing deep neural networks to be run on devices even powered by coin batteries, meaning that low-power microcontrollers can make sense of sensor data in much more sophisticated ways than previously possible.

The most prominent development in this field has been TensorFlow Lite for Microcontrollers, which is a Python framework specifically made for running machine learning models on microcontrollers.
The original Tensorflow has been an extremely popular machine learning framework for over a decade, but the extension to make it work effectively on microcontrollers has only been developed in recent years.
Tensorflow, as well as Tensorflow Lite for Microcontrollers, are also specifically designed for developers to work with neural networks rather than other types of machine learning models.

\subsection{Hand Gesture Data}\label{subsec:hand-gesture-data}
In this research, neural networks are used to detect gestures, but to do this, data from some sensor(s) must be fed into the network.
Based on this data alone, it must be possible to distinguish which gesture was performed for the neural network to achieve an acceptable classification accuracy.

There are a variety of sensors which could be used to record hand gestures and provide this data.
One of these is an accelerometer, which lies on the user's wrist (typically from a smartwatch) to track the direction and acceleration of the their hand movements.
Another option is using a camera to record the user's hand, which provides a huge amount of information but is susceptible to noise and requires significantly more intensive processing.
Photodiodes are sensors which output a signal which increases with the amount of light that hits them.
This means they can exploit ambient light by tracking shadows cast by the user's hand, therefore making it possible to recognize which gesture is being performed.
This project uses photodiodes for their much lower cost compared to cameras as well as the fact that they don't require the user to wear a device on their wrists, unlike accelerometers.
