\section{Model Design}\label{sec:model-design}
\subsection{Neural Network Architectures Tested}\label{subsec:architectures-tested}
Recognizing hand gestures based on photodiode data can be thought of as a time-series classification task, which is a type of problem that recurrent neural networks (RNNs) are especially well suited for~\cite{HUSKEN2003223}.
RNNs differ from conventional neural networks as they do not process the input in its entirety, but instead process each time step, or sample, from the input sequentially.
Each time step is used to update a "hidden state", which is fed back into the RNN along with the next time step.
This makes RNNs suitable for operating on sequences, as the order of the data is considered as well as the content of the data, unlike typical ANNs.
Thanks to this desirable property, RNNs were the first type of neural network tested.

Although RNNs are highly suitable for time-series classification, they suffer from the "vanishing gradient problem", which has since been overcome by long short-term memory cells (LSTMs) and gated recurrent units (GRUs)~\cite{DBLP:journals/corr/abs-1801-06105}.
To briefly explain this problem, time steps further in the past tend to have an exponentially lower weight in determining the hidden state, meaning that only the past few time steps are actually represented in the hidden state.
This means that in practice, RNNs tend to ignore if earlier time steps are out of order, which limits performance for long data sequences.
LSTMs and GRUs solve this issue using so-called "gates", which determine which contents of the previous hidden state should be kept and which contents of the current hidden state should be propagated to the next time step.
This largely mitigates the vanishing gradient problem as only relevant information is kept, therefore greatly reducing the rate at which hidden state weights decay.
Given that LSTMs and GRUs are should yield better classification performance than RNNs due to this advantage, these were also investigated.

One issue of using LSTMs and GRUs, however, is that they only accept a single data sequence with multiple features as input, in which each time step contains a single value from each feature, i.e.\ a 1D array.
This is relevant as with 3D-formatted data, each time step is a 2D frame - not a 1D array.
Therefore, each frame has to be flattened first, which causes a loss of spacial information.
This issue can be solved by using a convolutional neural network (CNN) to first extract features from the input data, which are then fed into the LSTM~\cite{KIM201972}.
CNNs accept 2D images as input, which means that frames do not have to be flattened.
By using convolutional and pooling layers, the resolution of this image can be reduced to just a single value for each neuron in the final layer of the CNN\@.
Therefore, the final CNN layer effectively outputs a 1D array with a size equal to the number of neurons in it, which can then be used as input to the LSTM with no loss in spacial information.
Due to this, the CNN-LSTM architecture was also investigated in this study.

Finally, transformer encoders were also tested for this project as a novel alternative to RNN based architectures.
Transformers use a concept called self-attention which identifies which elements in a sequence carry most semantic value based on the other elements of the sequence~\cite{https://doi.org/10.48550/arxiv.1706.03762}
This is fundamentally different to the hidden state mechanism used by RNNs and their derivatives.
They are typically used in language translation and are made up of two parts, an encoder and a decoder.
The encoder converts the input into an internal representation, which can then be manipulated and decoded into a more useful format, while the decoder converts this internal representation back into a more useful data type.
In a study by Y{\"u}ksel \textit{et al.}~\cite{yuksel-etal-2019-turkish}, a transformer encoder was used to classify topics from sequences of Turkish text taken from Twitter.
Although this is clearly different to the goal presented in this paper, i.e.\ classifying gestures based on photodiode readings, both are classification tasks using input data in which order is important.
Given the positive results found in this study, it seemed reasonable to test transformer encoders for gesture recognition as well.

\subsection{Parameter Tuning}\label{subsec:frame-size}
When training a machine learning model, the model parameters can have a substantial impact on final performance.
For the neural networks tested in this paper, some of these parameters affect almost all model types, while some are specific to individual architectures.
These are outlined below:

\subsubsection{All Architectures}
\begin{itemize}
    \item Frame size
    \item Number of layers
    \item Number of neurons per layer
\end{itemize}

\subsubsection{CNN+LSTM}
\begin{itemize}
    \item Number of convolutional layers
    \item Number of neurons per convolutional layer
    \item Filter kernel sizes
\end{itemize}

\subsubsection{Transformer Encoder}
\begin{itemize}
    \item Number of attention heads
\end{itemize}

In reality, there are more parameters that could be tuned, but the ones listed above are likely to have the largest impact on final model
Default values are used for parameters that are not mentioned in this section.
This paper does not go into detail regarding the testing of different parameters to determine which combinations yield optimal performance, but the best performing parameter values found are stated in section~\ref{subsec:accuracy-testing}.
